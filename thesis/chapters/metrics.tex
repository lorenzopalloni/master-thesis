\myChapter{Metrics}
\label{chap:Metrics}

To evaluate the performance of deep learning models for super-resolution, various metrics have been proposed in the current literature, which can be broadly categorized into traditional and perceptual metrics.

\section{Traditional Metrics}
\label{sec:traditional-metrics}
Traditional metrics are based on simple numerical comparisons between the generated and ground-truth images. Some commonly used traditional metrics include PSNR, MSE, SSIM \cite{wang2004image}, and MS-SSIM \cite{wang2003multiscale}.

PSNR (Peak Signal-to-Noise Ratio) is a widely used metric that measures the ratio of the peak signal power to the noise power in an image. It is calculated as the logarithm of the ratio of the maximum possible pixel value to the mean squared error between the predicted and ground-truth images. However, PSNR has been criticized for not being a reliable measure of image quality, as it does not correlate well with human perception.

MSE (Mean Squared Error) measures the average squared difference between the predicted and ground-truth images, with lower values indicating better image quality. However, like PSNR, it has been found to poorly correlate with human perception.

SSIM (Structural SIMilarity) is a more sophisticated metric that takes into account both structural information and pixel values in the image. It measures the similarity between the predicted and ground-truth images based on their luminance, contrast, and structure, and has been found to better correlate with human perception than PSNR and MSE.

MS-SSIM (Multi-Scale SSIM) is an improved version of the SSIM metric that takes into account the multi-scale nature of the human visual system.

\section{Perceptual Metrics}
\label{sec:perceptual-metrics}
Perceptual metrics aim to evaluate image quality based on human perception, measuring the visual similarity between predicted and ground-truth images, rather than simply their pixel-wise differences. Examples of deep learning-based perceptual metrics include FID \cite{heusel2017gans}, LPIPS \cite{zhang2018unreasonable}, LPIPS-Comp \cite{patel2021saliency}, E-LPIPS \cite{kettunen2019lpips}, and DISTS \cite{ding2020image}.

FID (Fréchet Inception Distance) is a perceptual metric used to evaluate the similarity between two sets of images by measuring the distance between their feature representations obtained from a pre-trained neural network.

LPIPS (Learned Perceptual Image Patch Similarity) computes the similarity between two images based on their perceptual similarity at the patch level, using a deep neural network trained on human perceptual judgments.

LPIPS-Comp (LPIPS with Saliency Map Comparison) is an extension of LPIPS that incorporates saliency maps to enhance the metric sensitivity to the most salient regions in the image.

E-LPIPS (Ensembled LPIPS) is an improved version of the LPIPS metric that employs an ensemble of neural networks trained on different subsets of images to improve the stability and robustness of the metric.

DISTS (Deep Image Structure and Texture Similarity) is based on a Siamese neural network, which takes two input images, and extracts features from them. These features are then compared at multiple levels to compute a final similarity score between the two images.

Other perceptual metrics such as MOS (Mean Opinion Score), 2AFC (Two Alternative Forced Choice), and JND (Just Noticeable Difference) are all examples of perceptual metrics that are based on subjective human evaluations of image quality.

MOS involves asking human subjects to rate the quality of the predicted images on a scale from 1 to 5, with the MOS score calculated as the average of these ratings. 2AFC involves presenting two images to human subjects and asking them to choose the one that appears to be of higher quality, while JND involves asking subjects to identify the minimum perceptible difference between two images.

Overall, the choice of metric for evaluating super-resolution models depends on the specific application and the goals of the study. Traditional metrics such as PSNR, MSE, and SSIM are simple to compute and provide a good baseline for comparison. Perceptual metrics such as LPIPS and MOS provide a more accurate measure of human perception but are more complex to compute and require additional resources. A combination of both traditional and perceptual metrics can provide a comprehensive evaluation of the performance of super-resolution models.

While Full-Reference IQA (FR-IQA) measures have been discussed thus far, it is worth noting that No-Reference IQA (NR-IQA) metrics are also commonly employed to assess processed images without their original counterparts. Examples of such NR-IQA metrics include BRISQUE, NIQE, PIQE, and CONTRIQUE. Datasets like LIVE, TID2008, CSIQ, and TID2013 are considered Full-Reference IQA (FR-IQA) datasets since they need a reference image to assess the image quality. In contrast, AVA and LIVE In the Wild are No-Reference IQA (NR-IQA) datasets, where image quality is evaluated independently without the necessity of a reference image.

\section{VMAF}

Video Multimethod Assessment Fusion (VMAF) is a perceptual video quality metric developed by Netflix to evaluate the visual quality of compressed videos. VMAF combines multiple elementary quality metrics to provide a single score that reflects human perception of video quality. It is trained using machine learning techniques to model the behavior of the human visual system (HVS).

Here's an outline of the math and concepts behind VMAF:

    Feature extraction: VMAF extracts several low-level features from both the original (reference) video and the compressed (distorted) video. These features are designed to capture different aspects of video quality as perceived by the human visual system. The primary features include:
        VIF (Visual Information Fidelity): Measures the amount of shared visual information between the reference and distorted videos.
        SSIM (Structural Similarity Index): Evaluates the structural distortions in the distorted video compared to the reference video.
        ADM (Adaptive Deviation Metric): Measures the deviation in local luminance and contrast statistics.
        Motion: Estimates the amount of motion in the video, as motion impacts the visibility of compression artifacts.

    Feature pooling: VMAF divides the video frames into non-overlapping blocks and computes the elementary metrics (e.g., VIF, SSIM, ADM) for each block. Then, it applies a pooling strategy (e.g., mean or harmonic mean) to aggregate the scores across all blocks in a frame. This results in a per-frame score for each elementary metric.

    Support Vector Regression (SVR) model: VMAF uses a support vector regression model to combine the per-frame elementary metrics into a single quality score. The model is trained on a dataset of videos annotated with subjective quality scores, which represent the opinions of human observers. The SVR model learns to predict these subjective scores based on the elementary metrics.

    Final VMAF score: The output of the SVR model is a per-frame VMAF score that reflects the perceived video quality. These per-frame scores are usually pooled (e.g., using the mean) to compute a single VMAF score for the entire video.

In summary, VMAF combines multiple elementary quality metrics, aggregates the results using pooling strategies, and employs a machine learning model to predict human perception of video quality. The math behind VMAF involves feature extraction, pooling, and support vector regression modeling to provide a single, meaningful quality score.

% These perceptual metrics are commonly used in the evaluation of super-resolution models, in addition to deep learning-based perceptual metrics. MOS is often used as a benchmark for evaluating the visual quality of predicted images, while 2AFC and JND are used to measure the difference in visual quality between predicted and ground-truth images. Together, these metrics provide a comprehensive evaluation of the visual quality of predicted images from different perspectives.

% Another approach that enphasizes the importance of edges in restored images is the ERQA metric, introduced in \cite{kirillova2021erqa}, and improved in \cite{lyapustin2022towards}, that is based on the Canny edge detector algorithm.

% Super-resolution (SR) with deep learning models has achieved impressive quantitative performance in terms of peak signal-to-noise ratio (PSNR) and structural similarity (SSIM) metrics. However, improving the perceptual quality of the generated images remains a challenge, especially when using quantitative metrics such as Learned Perceptual Image Patch Similarity (LPIPS).
% 
% One of the main challenges in improving the perceptual quality of SR images is to preserve fine-grained details while avoiding artifacts and noise amplification. Deep learning models for SR can generate images with high-frequency details, but these details may not be consistent with the true high-resolution images, leading to a loss of perceptual quality. This can be exacerbated when training datasets do not contain enough diverse and realistic high-resolution images.
% 
% Another challenge in SR with deep learning models is the balance between overfitting and underfitting. Overfitting occurs when the model memorizes the training data and fails to generalize to new data, resulting in poor performance on the validation or test sets. On the other hand, underfitting occurs when the model is too simple to capture the complex mapping between low-resolution and high-resolution images, resulting in poor performance on both the training and validation sets.
% 
% In addition to these challenges, the use of quantitative metrics such as LPIPS to evaluate the perceptual quality of SR images has its limitations. LPIPS is a learned distance metric that measures the perceptual difference between two images based on the feature representations of a pre-trained deep neural network. While LPIPS can capture some aspects of human perception, it is not a perfect measure of image quality and may not fully capture all the nuances of perceptual quality.
% 
% Despite these challenges, ongoing research is addressing these issues in SR with deep learning models. For example, recent studies have proposed novel loss functions and network architectures to improve the perceptual quality of SR images, while others have explored more diverse and realistic training datasets to improve generalization. In addition, alternative evaluation metrics such as human perception studies can provide a more comprehensive assessment of perceptual quality.

% This chapter investigates various Image Quality Assessment (IQA) metrics commonly used in the current literature to evaluate results and improve performance of super-resolution and compression artifact removal operations.

% For our purpose, a metric is as good as how much it agrees with the average human judgement. The agreement is often measured with Pearson, Kendall, and Spearman correlation coefficients, while the average human judgement is estimated using sets of images structured for Mean Opinion Score (MOS), Two Alternative Force Choices (2AFC), or Just Noticeable Difference (JND) approaches.


%%%%%%%%%%%%%%%%%%%%%%%%%%%%%%%%%%%%%%%%%%%%%%%%%%%%%%%%%%%%%%%%%%%%%%%%%%%%%%%%%%%%%%%%
% \label{sec:psnr}
Let $y$ be an image and $\hat{y}$ be a distorted version of $y$.
A measure of similarity between the two given images is the Peak Signal-to-Noise Ratio:

\begin{align}
    PSNR(y, \hat{y}) \coloneqq 10 \cdot \log_{10} \left\{ \frac{MAX(y)^2}{MSE(y, \hat{y})} \right\},
\end{align}

where $MAX(y)$ is the maximum value of $y$, and $MSE(y, \hat{y}) \coloneqq || y - \hat{y} ||^2_2$.

% \section{SSIM}
\label{sec:ssim}

Let $y = \left\{ y_i | i = 1, 2, \dots, N \right\}$ and $\hat{y} = \left\{ \hat{y}_i | i = 1, 2, \dots, N \right\}$ be two discrete non-negative signals that have been aligned with each other, and let $\mu_y$, $\sigma_y$ and $\sigma_{y\hat{y}}$ be the mean of $y$, the variance of $y$, and the covariance of $y$ and $\hat{y}$, respectively. Approximately, $\mu_y$ and $\sigma_y$ can be viewed as estimates of the luminance and contrast of $x$, and $\sigma_{y\hat{y}}$ measures the tendency of $y$ and $\hat{y}$ to vary together, thus an indication of structural similarity. In \cite{wang2004image}, the luminance, contrast and structure comparison measures were given as follows:

\begin{align}
l(y, \hat{y}) = \frac{2 \mu_y \mu_{\hat{y}} + C_1}{\mu^2_y + \mu^2_{\hat{y}} + C_1}\label{eq:luminance} \\
c(y, \hat{y}) = \frac{2\sigma_y\sigma_{\hat{y}} + C_2}{\sigma^2_y + \sigma^2_{\hat{y}} + C_2}\label{eq:constrast} \\
s(y, \hat{y}) = \frac{\sigma_{y\hat{y}} + C_3}{\sigma_y \sigma_{\hat{y}} + C_3}\label{eq:structure}
\end{align}

where $C_1$, $C_2$, $C_3$ are small constants given by
\begin{align}
C_1 = \left( K_1 L \right)^2, C_2 = \left( K_2 L \right)^2 \text{ and } C_3 = C_2 / 2,
\end{align}
respectively. $L$ is the dynamic range of the pixel values (255 for 8-bit grey-scale images), and $K_1 \ll 1, K_2 \ll 1$ are small scalar constants. In \cite{wang2004image} $K_1$ and $K_2$ are set to $0.01$ and $0.03$, respectively. The general form of the Structural SIMilarity index between signal $y$ and $\hat{y}$ is defined as:

\begin{align}
SSIM(y, \hat{y}) = [l(y, \hat{y})]^{\alpha} [c(y, \hat{y}]^{\beta} [s(y, \hat{y}]^{\gamma},
\end{align}
where $\alpha$, $\beta$ and $\gamma$ are parameters to define the relative importance of the three components. Specifically, we set $\alpha = \beta = \gamma = 1$, and the resulting SSIM index is given by

\begin{align}
SSIM(y, \hat{y}) = \frac{
    \left( 2 \mu_{y} \mu_{\hat{y}} + C_1 \right) \left( 2 \sigma_{y \hat{y}} + C_2 \right)
}{
    \left( \mu_{y}^2 + \mu_{\hat{y}}^2 + C_1 \right) \left( \sigma_{y}^2 \sigma_{\hat{y}}^2 + C_2 \right)
}
\end{align}

which satisfies the following conditions:
\begin{enumerate}
\item symmetry: $SSIM(y, \hat{y}) = SSIM(\hat{y}, y)$;
\item boundedness: $SSIM(y, \hat{y}) \leq 1$;
\item unique maximum: $SSIM(y, \hat{y}) = 1 \iff y = \hat{y}$.
\end{enumerate}

In practise the SSIM index is applied locally, using an $11\times11$ circular-symmetric Gaussian weighting function $w = \{w_i | i = 1, 2, \dots, N\}$, with a standard deviation of $1.5$ samples, normalized to unit sum $\left( \sum^{N}_{i=1} w_i = 1 \right)$.
\begin{align}
    MSSIM(y, \hat{y}) =  \frac{1}{M} \sum^{M}_{j=1} SSIM(y_j, \hat{y}_j)
\end{align}
where $y_j$ and $\hat{y}_j$ are the image contents at the $j$-th local windows, and $M$ is the number of local windows of the image.

The local statistics should be modified according to $w$:
\begin{itemize}
\item $ \mu_{y} = \sum^{N}_{i = 1} w_i y_i $
\item $ \sigma_{y} = \left( \sum^{N}_{i = 1} w_i (y_i - \mu_y)^2 \right) ^ {\frac{1}{2}} $
\item $ \sigma_{y \hat{y}} = \sum^{N}_{i = 1} w_i (y_i - \mu_y) (\hat{y}_i - \mu_{\hat{y}}) $.
\end{itemize}

% \section{MS-SSIM}
\label{sec:msssim}
A single-scale method as described in the previous section may be appropriate only for specific settings. The multi-scale method is a convenient way to incorporate image details at different resolutions. The authors of \cite{wang2003multiscale} proposed a multi-scale SSIM method that taking the reference and distorted image signals as the input, the system iteratively applies a low-pass filter and down-samples the filtered image by a factor of 2. The original image has been indexed as Scale $1$1, while the highest scale as Scale $M$, which is obtained after $M - 1$ iterations. At the $j$-th scale, the contrast comparison \ref{eq:constrast} and the structure comparison \ref{eq:structure} are calculated and denoted as $c_j (x, y)$ and $s_j (x, y)$, respectively. The luminance comparison \ref{eq:luminance} is computed only at Scale $M$ and is denoted as $l_M (x, y)$. The overall SSIM evaluation is obtained by combining the measurement at different scales using

\begin{align}
SSIM(y, \hat{y}) = [l_M(y, \hat{y})]^{\alpha_M} \cdot \prod^{M}_{j = 1} [c_j(y, \hat{y})]^{\beta_j} [s_j(y, \hat{y})]^{\gamma_j}
\end{align}

To calibrate the system parameters, the authors of \cite{wang2003multiscale} involved 8 subjects in their experiments, asking to assess the quality of the synthesized images. With this approach the researcher estimated $\beta_1 = \gamma_1 = 0.0448, \beta_2 = \gamma_2 = 0.2856, \beta_3 = \gamma_3= 0.3001, \beta_4 = \gamma_4 = 0.2363, \text{ and } \alpha_5 = \beta_5 = \gamma_5 = 0.1333$, respectively.

% \section{LPIPS-Comp}
\label{sec:lpipscomp}
The same technique used to train LPIPS has been adopted for LPIPS-Comp. While LPIPS is trained on BAPPS, that contains images with several distortions but accounts only for compression artefacts from JPEG, LPIPS-Comp has seen a compression specific perceptual similarity dataset. In doing so, experiments showed that LPIPS-Comp aligns more to human judgement than the standard LPIPS on general compression tasks.

LPIPS-Comp \cite{patel2021saliency} is a perceptual similarity metric based on deep neural networks obtained following the same approach as in \cite{zhang2018unreasonable} with LPIPS.
These methods employ the $ReLU$ activations after each \textit{conv} block in the first five layers of the VGG-16 \cite{simonyan2014very} architecture, with batch-normalization \cite{ioffe2015batch}.

Feed-forward is performed on VGG-16 for both the original ($y$) and the reconstructed image ($\hat{y}$).
Let $L$ be the set of layers used for loss calculation (five for our setup), a function $F(y)$ denoting feed-forward on an input image $y$.
$F(y)$ and $F(\hat{y})$ return two stacks of feature activation’s for all $L$ layers.

The LPIPS-Comp loss is then computed as:
\begin{itemize}
\item $F(y)$ and $F(\hat{y})$ are unit-normalized in the channel dimension. Let us call these, $z^l_y, z^l_{\hat{y}} \in R^{H_l \times W_l \times C_l}$ where $l \in L$. ($H_l, W_l$ are the spatial dimensions).
\item $z^l_y , z^l_{\hat{y}}$ are scaled channel wise by multiplying with the vector $w_l \in R^{C_l}$.
\item The $L_2$ distance is then computed and an average over spatial dimension is taken. Finally, a channel-wise sum is performed.
\end{itemize}

Equation \ref{eq:lpips-comp} summarizes the LPIPS-Comp computation.

\begin{align}
\text{LPIPS-Comp}(y, \hat{y}) = \sum_{l} \frac{1}{H_l W_l} \sum_{h, w} \left| \left| w_l \odot \left( z^l_{\hat{y}, h, w} - z^l_{y, h, w} \right) \right| \right|^2_2\label{eq:lpips-comp}
\end{align}

Note that the weights in $F$ are learned for image classification on the ImageNet dataset \cite{russakovsky2015imagenet} and are kept fixed. $w$ are the linear weights learned on top of $F$ on the BAPPS \cite{zhang2018unreasonable} dataset for LPIPS and on a compression specific similarity dataset for LPIPS-Comp. While in the first, compressed images are obtained with JPEG only, the second makes use of several compression methods: Mentzer et al. \cite{mentzer2018conditional}, Patel et al. \cite{patel2019deep}, BPG \cite{bellard2014bpg} and JPEG-2000 \cite{skodras2001jpeg}.

%%%%%%%%%%%%%%%%%%%%%%%%%%%%%%%%%%%%%%%%%%%%%%%%%%%%%%%%%%%%%%%%%%%%%%%%%%%%%%%%%%%%%%%%

% \section{DISTS}
% \label{sec:dists}
% The authors of DISTS \cite{} carried out five major experiments. First, they showed that DISTS has not the best performance overall on LIVE \cite{}, CSIQ \cite{}, and TID2013 \cite{} that are datasets that have been around in the literature long enough to be likely overfitted by recent quality measures. Second, they saw comparable results on the BAPPS dataset against LPIPS (that is a metric trained on the BAPPS dataset). Third, DISTS achieves best performance...

% \section{SFSN}
% \label{sec:sfsn}
% The authors of \cite{zhou2021image} found that a linear combination of a local structural fidelity assessment (SF) and a global statistical naturalness measure (SN) achieves high correlation with human judgement (measured with MOS on public Single Image Super Resolution IQA datasets, such as WIND \cite{yeganeh2015objective}, CVIU \cite{ma2017learning}, and QADS \cite{zhou2019visual}).

%%%%%%%%%%%%%%%%%%%%%%%%%%%%%%%%%%%%%%%%%%%%%%%%%%%%%%%%%%%%%%%%%%%%%%%%%%%%%%%%%%%%%%%%%%%%%%%%%%%%
% \begin{algorithm}
% \caption{SFSN}
% \textbf{Input:} original image $y$; distorted image $\hat{y}$.\\
% \textbf{Output:} SFSN quality score.\\
% \label{alg-kcv}
% \begin{algorithmic}
% 
% \State Convert both $y$ and $\hat{y}$ to grayscale
% \State For efficiency it is possible to scale both images within $[0, 1]$
% \State Get $m$ and $n$, that are respectively the height and the width of $\hat{y}$
% \State $s$ = min(m, n)
% \State Resize $y$ to $s \times s$ $\rightarrow$ $y^{(1)}$
% \State Apply 2D discrete cosine transform to $y^{(1)}$ $\rightarrow$ $y^{(2)}$
% \State Split $y_2$ in low and high frequency $\rightarrow$ $y^{3}_{high}$, $y^{3}_{low}$
% \State Apply the inverse of the 2D discrete cosine transform $\rightarrow$ $y^{4}_{high}$, $y^{4}_{low}$
% 
% \State Resize $\hat{y}$ to $s \times s$ $\rightarrow$ $\hat{y}^{(1)}$
% \State Apply 2D discrete cosine transform to $\hat{y}^{(1)}$ $\rightarrow$ $\hat{y}^{(2)}$
% \State Split $\hat{y}_2$ in low and high frequency $\rightarrow$ $\hat{y}^{3}_{high}$, $\hat{y}^{3}_{low}$
% \State Apply the inverse of the 2D discrete cosine transform $\rightarrow$ $\hat{y}^{4}_{high}$, $\hat{y}^{4}_{low}$
% 
% \State SF = MS-SSIM($y^{(4)_{low}}$, $\hat{y}^{4}_{low}$)
% \State SN = Entropy($y^{(4)}_{high}$)
% 
% \State \Return $(0.9) * SF + 0.1 * SN$  \\ (1 + 0.9) * SF + (1 - 0.9) * SN
% 
% \end{algorithmic}
% \end{algorithm}



%%%%%%%%%%%%%%%%%%%%%%%%%%%%%%%%%%%%%%%%%%%%%%%%%%%%%%%%%%%%%%%%%%%%%%%%%%%%%%%%%%%%%%%%
% from DISTS paper
% For more than 50 years, the mean squared error (MSE)
% was the standard full-reference method for assessing signal
% fidelity and quality, and it continues to play a fundamental
% role in the development of signal and image processing
% algorithms, despite its poor correlation with human percep-
% tion [1], [2].
% [1] -> Z. Wang and A. C. Bovik, “Mean squared error: Love it or leave
% it? A new look at signal fidelity measures,” IEEE Signal Processing
% Magazine, vol. 26, no. 1, pp. 98–117, 2009.
% 
% [2] -> B. Girod, “What’s wrong with mean-squared error,” in Digital
% Images and Human Vision, A. B. Watson, Ed. The MIT Press, 1993,
% pp. 207–220
%%%%%%%%%%%%%%%%%%%%%%%%%%%%%%%%%%%%%%%%%%%%%%%%%%%%%%%%%%%%%%%%%%%%%%%%%%%%%%%%%%%%%%%%

% - ERQAv1  % in this paper there is an interesting barplot of the most used QA metrics
%     - Edge-Restoration Quality Assessment for Video Super-Resolution
% - ERQAv2
%
% - VMAF
% 
% % language based image quality assessment
% - In the article "Language Based Image Quality Assessment", the authors claim that a fine grained semantic computer vision task can be a great proxy for human level image judgement.
% 
% - NoGAN approach
%     - Image and Video Restoration and Compression Artefact Removal Using a NoGAN Approach 
%     - method developed in DeOldify, then used in https://www.fast.ai/2019/05/03/decrappify/ 
%     - NoGAN is a method to train a GAN architecture to obtain better
%         results and stabilizing training and generation of images. The
%         main idea is to pre-train generator and discriminator separately
%         and then perform a final adversarial training step as is performed in
%         standard GANs. In this setting the generator is initially trained using
%         some perceptual loss, then the generated “fake" images are used
%         to train the discriminator as a binary classifier.
%     - Both full-reference
%         image quality metrics (i.e. metrics that compare a processed image
%         to the original high-quality image – SSIM [ 13] and LPIPS [15]) and
%         no-reference metrics (i.e. metrics that evaluate the naturalness of
%         an image – BRISQUE [11] and NIQE [12 ]) have been used to eval-
%         uate the performance of the system and the effectiveness of the
%         perceptual loss and of the final GAN training step.
% 
% - U-Net
% - SR-UNet
%     - can be used to perform super resolution and compression artifact removal in videos.
%     - effectiveness of SR-UNet demostrated using:
%         - signal-based scores such as VMAF
%         - perceptual-based scores such as LPIPS
% 
