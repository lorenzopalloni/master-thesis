\myChapter{Background}
\label{chap:Background}

Video restoration is an essential task in the field of computer vision and image processing. The primary objective of video restoration is to improve the visual quality of degraded videos caused by various factors such as noise, blur, compression artifacts, and other distortions. In recent years, deep learning techniques have gained significant attention in the field of video restoration due to their superior performance compared to traditional methods. In this chapter, we will review the existing literature on video restoration, focusing on the optimization techniques of deep learning models for visual quality improvement.

Traditional video restoration techniques involve the use of mathematical algorithms to reduce the effects of noise, blur, and other distortions. Some of the commonly used techniques include median filtering, bilateral filtering, and wavelet-based denoising. While these methods are effective in restoring some aspects of the video quality, they have limited performance in handling complex distortions and restoring high-frequency details.

Deep learning-based video restoration methods have been shown to achieve state-of-the-art performance in restoring degraded videos. These methods use deep neural networks to learn the underlying mapping between the degraded and the ground-truth images. The use of deep learning techniques allows for the restoration of high-frequency details and the removal of complex distortions in videos.

Super-resolution is the process of increasing the spatial resolution of an image or video. It is a critical task in video restoration as it helps to improve the visual quality of videos. Super-resolution can be achieved using both traditional and deep learning-based techniques. However, deep learning-based techniques have shown superior performance in recent years.

Deep learning-based super-resolution techniques typically use convolutional neural networks (CNNs) to learn the underlying mapping between low-resolution and high-resolution images. The most commonly used deep learning-based super-resolution techniques are single-image super-resolution (SISR) and multiple-image super-resolution (MISR).

In SISR, a CNN is trained to generate a high-resolution image from a single low-resolution image. The CNN learns the underlying mapping between low-resolution and high-resolution images by minimizing a loss function that measures the difference between the generated high-resolution image and the ground-truth high-resolution image.

In MISR, a CNN is trained to generate a high-resolution image from multiple low-resolution images. The CNN learns the underlying mapping between low-resolution and high-resolution images by using multiple input images to improve the accuracy of the generated high-resolution image.

One of the state-of-the-art super-resolution techniques is the Residual Dense Network (RDN). RDN is a deep learning-based super-resolution technique that uses residual dense blocks (RDBs) to learn the underlying mapping between low-resolution and high-resolution images. RDN has shown superior performance compared to other super-resolution techniques in terms of both quantitative and qualitative measures.

Another state-of-the-art super-resolution technique is the Recursive Spatially Variant Deconvolutional Network (RSVDecon). RSVDecon is a deep learning-based super-resolution technique that uses a recursive spatially variant deconvolutional network to generate high-resolution images from low-resolution images. RSVDecon has shown superior performance in restoring high-frequency details in videos.

The RSGAN model is another state-of-the-art super-resolution technique that combines generative adversarial networks (GANs) with residual learning. The RSGAN model uses a generator network to generate high-resolution images from low-resolution images and a discriminator network to distinguish between the generated high-resolution images and the ground-truth high-resolution images. RSGAN has shown superior performance in restoring high-frequency details and natural textures in videos.

