\myChapter{Background}
\label{chap:Background}

Video restoration is an essential task in the field of computer vision and image processing. The primary objective of video restoration is to improve the visual quality of degraded videos caused by various factors such as noise, blur, compression artefacts, and other distortions. In recent years, deep learning techniques have gained significant attention in the field of video restoration due to their superior performance compared to traditional methods. In this chapter, we will review the existing literature on video restoration, focusing on the optimization techniques of deep learning models for visual quality improvement.

Traditional video restoration techniques involve the use of mathematical algorithms to reduce the effects of noise, blur, and other distortions. Some of the commonly used techniques include median filtering, bilateral filtering, and wavelet-based denoising. While these methods are effective in restoring some aspects of the video quality, they have limited performance in handling complex distortions and restoring high-frequency details.

Deep learning-based video restoration methods have been shown to achieve state-of-the-art performance in restoring degraded videos. These methods use deep neural networks to learn the underlying mapping between the degraded and the ground-truth images. The use of deep learning techniques allows for the restoration of high-frequency details and the removal of complex distortions in videos.

Super-resolution is the process of increasing the spatial resolution of an image or video. It is a critical task in video restoration as it helps to improve the visual quality of videos. Deep learning-based super-resolution techniques typically use convolutional neural networks (CNNs) to learn the underlying mapping between low-resolution and high-resolution images.

The first widely adopted CNN-based SR model was the Super-Resolution Convolutional Neural Network (SRCNN) proposed by Dong et al. (2014). SRCNN utilizes a three-layer CNN architecture with patch-based training to learn an end-to-end from low-resolution to high-resolution images. Since then, many variants of CNN-based SR models have been proposed, including VDSR (Kim et al., 2016), EDSR (Lim et al., 2017), and RCAN (Zhang et al., 2018), among others. These models typically have deeper and wider architectures than SRCNN and incorporate various techniques such as residual connections, feature normalization, and attention mechanisms to improve SR performance.
 
GAN-based SR models have also gained popularity in recent years. SRGAN (Ledig et al., 2017) is a notable example that utilizes a GAN architecture to generate high-resolution images from low-resolution inputs. The generator network of SRGAN is trained to produce high-resolution images that are perceptually similar to the ground truth images, while the discriminator network is trained to distinguish between the generated and ground truth images. Since then, several GAN-based SR models have been proposed, including ESRGAN (Wang et al., 2018) and SRFBN (Liu et al., 2019).

% The most commonly used approaches are single-image super-resolution (SISR) and multiple-image super-resolution (MISR). In SISR, a CNN is trained to generate a high-resolution image from a single low-resolution image. The CNN learns the underlying mapping between low-resolution and high-resolution images by minimizing a loss function that measures the difference between the generated high-resolution image and the ground-truth high-resolution image. In MISR, a CNN is trained to genereate a high-resolution image from multiple low-resolution images. The CNN learns by using multiple input images to improve its accuracy in generating the high-resolution image. All experiments conducted in this research use models trained with the SISR approach.

