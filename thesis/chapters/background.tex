\myChapter{Background}
\label{chap:Background}

In this chapter, the existing literature on video restoration is reviewed, with a focus on optimization techniques for deep learning models aimed at enhancing visual quality.

% The first widely adopted CNN-based SR model was the Super-Resolution Convolutional Neural Network (SRCNN) proposed by Dong et al. (2014). SRCNN utilizes a three-layer CNN architecture with patch-based training to learn an end-to-end from low-resolution to high-resolution images. Since then, many variants of CNN-based SR models have been proposed, including VDSR (Kim et al., 2016), EDSR (Lim et al., 2017), and RCAN (Zhang et al., 2018), among others. These models typically have deeper and wider architectures than SRCNN and incorporate various techniques such as residual connections, feature normalization, and attention mechanisms to improve SR performance.
 
% GAN-based SR models have also gained popularity in recent years. SRGAN (Ledig et al., 2017) is a notable example that utilizes a GAN architecture to generate high-resolution images from low-resolution inputs. The generator network of SRGAN is trained to produce high-resolution images that are perceptually similar to the ground truth images, while the discriminator network is trained to distinguish between the generated and ground truth images. Since then, several GAN-based SR models have been proposed, including ESRGAN (Wang et al., 2018) and SRFBN (Liu et al., 2019).

