\documentclass{beamer}

\mode<presentation> {
    \usetheme{Madrid}
%     \usetheme{default}
%     \usetheme{PaloAlto}
%     \usetheme{EastLansing}
}

\beamertemplatenavigationsymbolsempty

% pacman.sty does not allow numbering pages properly with backup/appendix slides
% \usepackage{pacman}

\usepackage{graphicx} % Allows to include figures
% \usepackage{booktabs} % Allows the use of \toprule, \midrule and \bottomrule in tables
% \usepackage[utf8]{inputenc}
% \usepackage{float}
% \usepackage{mathtools}
% \usepackage{xcolor}
% \usepackage{bm}
% \usepackage{scalerel}
% \usepackage{setspace}
\usepackage{appendixnumberbeamer} % add \appendix

% \usepackage{amsmath}
% \usepackage{amssymb}
\usepackage{mathtools, nccmath}
% \usepackage[mathscr]{euscript}  % euler script font
% \usepackage{cleveref}
% \usepackage{enumitem}

\usepackage{algorithm}
\usepackage[noend]{algpseudocode}

\DeclareMathOperator{\Var}{Var}
\DeclareMathOperator{\modexp}{mod\_exp}
\DeclareMathOperator{\decryptiontime}{decryption\_time}
\DeclareMathOperator{\extrareduction}{extra\ reduction}

%---------------------------------------------------------------------------------------
%	TITLE PAGE
%---------------------------------------------------------------------------------------

\title[DSP - 2021/22 - Lorenzo Palloni]{Timing Attacks against RSA}
\subtitle{(Data Security and Privacy)}
\author{Lorenzo Palloni}
\institute[]{
  University of Florence\\
  \medskip
  \textit{lorenzo.palloni@stud.unifi.it}
}
\date{\today}

\begin{document}

\begin{frame}
\titlepage % Print the title page as the first slide
\end{frame}

%---------------------------------------------------------------------------------------
%	PRESENTATION SLIDES
%---------------------------------------------------------------------------------------
%   TABLE OF CONTENTES
%---------------------------------------------------------------------------------------
%\begin{frame}
%\tableofcontents
%\end{frame}
%---------------------------------------------------------------------------------------
%---------------------------------------------------------------------------------------
\begin{frame}
\frametitle{Introduction}

\begin{itemize}
  \item Main project goal $\rightarrow$ overview of two major timing attacks:
  \begin{itemize}
    \item Kocher's timing attack (1996) \cite{bib:kocher};
    \item Brumley and Boneh's timing attack (2005) \cite{bib:openssl}.
  \end{itemize}
\end{itemize}

\begin{itemize}
  \item A timing attack is a type of side-channel attack.
  \item A side-channel attack exploits physical parameters, such as:
  \begin{itemize}
    \item execution time;
    \item electromagnetic emission;
    \item supply current.
  \end{itemize}
\end{itemize}

\end{frame}
%---------------------------------------------------------------------------------------
\begin{frame}
\frametitle{Square-and-multiply modular exponentiation}

\begin{algorithm}[H]
\caption{Square-and-multiply modular exponentiation algorithm.}\label{alg:one}
\begin{algorithmic}[1]
\Function{$\modexp$}{$y, x, n$}
  \Comment{Computes $y^x \bmod n$}
  \State $R \leftarrow 1$\;
  \For{$k \leftarrow 0, w - 1$}
    \State $R \leftarrow (R \cdot R) \bmod n$\;
    \If{ $\text{(the } $k$ \text{-th bit of } $x$ \text{) is }1$ }
      \State $R \leftarrow (R \cdot y) \bmod n$\;
    \EndIf
  \EndFor
  \State \Return $R$
\EndFunction
\end{algorithmic}
\end{algorithm}

$\modexp$ is a core operation in public-key cryptosystems, such as:
\begin{itemize}
  \item RSA;
  \item Diffie-Hellman key exchange.
\end{itemize}

\end{frame}
%---------------------------------------------------------------------------------------
\begin{frame}
\frametitle{Kocher's timing attack - assumptions}

Assume an RSA cryptosystem, with $D_k[x] = y^x \bmod n$.

Now, suppose that an attacker:

\begin{itemize}
  \item wants to retrieve the private exponent $x$;
  \item already knows the first $b$ exponent bits of $x$;
  \item can perform as many decryption operations as he wants;
  \item is able to measure $T := e + \sum_{i=0}^{k-1} t_i$, where:
  \begin{itemize}
    \item $t_i$ $\rightarrow$ $i$-th iteration required time for $\modexp(y, x, n)$;
    \item $y$ $\rightarrow$ any ciphertext;
    \item $e$ $\rightarrow$ overhead time.
  \end{itemize}
\end{itemize}

\end{frame}
%---------------------------------------------------------------------------------------
\begin{frame}
\frametitle{Kocher's timing attack - pseudocode}

\begin{algorithm}[H]
\caption{Kocher's timing attack}\label{alg:two}
\begin{algorithmic}[1]
  \State generate $s$ ciphertexts $\{ y_1, \dots, y_s \}$;
  \State guess the $b$-th exponent bit $d_b' := 0$;
  \State measure $T'_j = e + \sum_{i = 0}^{b - 1} t'_i, \hspace{1cm} \forall j \in \{ 1, \dots, s\}$;
  \State estimate $\Var(T - T')$;
  \State repeat from step 3. and step 4. with $d'_b := 1$;
  \State choose $d^*_b \in \{0, 1\}$ that minimizes $\Var(T - T')$;
  \State set $d_b \leftarrow d^*_b$;
  \State set $b \leftarrow b + 1$;
  \State repeat from step 2. until $b > k - 1$.
\end{algorithmic}
\end{algorithm}

In step 3., $T'$ is measured by running $\modexp(y, x_b, n)$, where:
\begin{itemize}
  \item $x_b := (d_0 d_1 \cdots d_{b-1} d'_b)_2$.
\end{itemize}

\end{frame}
%---------------------------------------------------------------------------------------
% \begin{frame}
% \frametitle{Kocher's timing attack - why it works}
%
% \begin{itemize}
%   \item
% \end{itemize}
%
% \end{frame}
%---------------------------------------------------------------------------------------
\begin{frame}
\frametitle{Brumley and Boneh's timing attack - assumptions}

Assume an RSA cryptosystem implemented with OpenSSL (0.9.7).

Let $n = pq$ be the public modulus, with $q < p$.

Now, suppose that an attacker:

\begin{itemize}
  \item wants to retrieve the private factor $q$;
  \item knows $i - 1$ bits of $q$: $\{q_0, q_1, \dots, q_{i - 1}\}$;
  \item starts to guess $q$ with $g$:
  \begin{itemize}
    \item[$\rightarrow$] $g_0 := q_0,\ g_1 := q_1,\ \dots,\ g_{i - 1} := q_{i - 1}$;
    \item[$\rightarrow$] $g_i := 0,\ g_{i + 1} := 0,\ \dots,\ g_{k - 1} := 0$;
  \end{itemize}
  \item can perform as many decryption operations as he wants;
  \item knows that his guess $g \in [2^{511}, 2^{512} - 1]$ \footnote{The public modulus $n$ in OpenSSL 0.9.7 has a 1024-bit binary representation.}.

\end{itemize}

\end{frame}
%---------------------------------------------------------------------------------------
\begin{frame}
\frametitle{Brumley and Boneh's timing attack - pseudocode}

\begin{algorithm}[H]
\caption{Brumley and Boneh's timing attack against OpenSSL}\label{alg:three}
\begin{algorithmic}[1]
  \State set $g' := g$, then $g'_i := 1$;
  \State compute $u_g = gR^{-1} \bmod n$, and $u_{g'} = g'R^{-1} \bmod n$;
  \State measure $t_1 = \decryptiontime(u_g)$, and $t_2 = \decryptiontime(u_{g'})$;
  \State compute $\Delta = \left| t_1 - t_2 \right|$;
  \State \Return $0$ if $\Delta$ is "large". Otherwise ($\Delta$ is "small"), \Return $1$.
\end{algorithmic}
\end{algorithm}

Note that in step 1., if $q_i = 1$:
\begin{itemize}
  \item then $g < g' < q$;
  \item otherwise, $g < q < g'$.
\end{itemize}

\end{frame}
%---------------------------------------------------------------------------------------
\begin{frame}
\frametitle{Brumley and Boneh's timing attack - why it works}

Techniques implemented in OpenSSL (0.9.7) to improve $\modexp$:
\begin{itemize}
  \item Chinese Remainder $\rightarrow$ exposes $q$ $\Rightarrow p = n / q \Rightarrow d = e^{-1} \bmod \phi(n)$ \footnote{$\phi(n) = \phi(p) \cdot \phi(q) = (p - 1) \cdot (q - 1).$};
  \item Sliding Windows \cite{bib:sliding} $\rightarrow$ many multiplications by $g$;
  \item Montgomery multiplication \cite{bib:montgomery} $\rightarrow$ more time required when $g < q$ \cite{bib:schindler};
  \item Karatsuba's algorithm $\rightarrow$ less time required when $g$ approaches $q$ from below.
\end{itemize}

\end{frame}
% %---------------------------------------------------------------------------------------
% \begin{frame}
% \frametitle{Brumley and Boneh's timing attack - simulations}
% 
% Time variations for each guessed bit ($32$-bits factor):
% \begin{itemize}
%   \item without blinding (left): almost deterministic;
%   \item with blinding (right): unpredictable.
% \end{itemize}
% 
% \begin{columns}[t]
%   \column{.5\textwidth}
%     \centering
%     \includegraphics[width=1.0\textwidth]{static/backup-attack_without_blinding}
%   \column{.5\textwidth}
%     \centering
%     \includegraphics[width=1.0\textwidth]{static/backup-attack_with_blinding}
% \end{columns}
% 
% \end{frame}
% %---------------------------------------------------------------------------------------
\begin{frame}
\frametitle{Conclusion}

In 1996, Kocher:
\begin{itemize}
  \item showed that simple $\modexp$ exposes the exponent;
  \item prompted improvements on $\modexp$ implementations.
\end{itemize}

In 2005, Brumley and Boneh:
\begin{itemize}
  \item proved that remote timing attacks are practical;
  \item made crypto libraries to implement blinding by default.
\end{itemize}

\end{frame}

%---------------------------------------------------------------------------------------
%---------------------------------------------------------------------------------------
%---------------------------------------------------------------------------------------
%---------------------------------------------------------------------------------------
%---------------------------------------------------------------------------------------
%---------------------------------------------------------------------------------------
\begin{frame}[c,noframenumbering]

  {\Huge \emph{Thanks for your attention!}}

  \begin{itemize}
    \item[]
    \item[]
  \end{itemize}

  \LARGE Do you have any questions?

\end{frame}
%---------------------------------------------------------------------------------------
\begingroup
\footnotesize
\begin{frame}[noframenumbering]
\frametitle{References}
\begin{thebibliography}{99}

\bibitem{bib:kocher}{Kocher, P.C., 1996, August. Timing attacks on implementations of Diffie-Hellman, RSA, DSS, and other systems. In Annual International Cryptology Conference (pp. 104-113). Springer, Berlin, Heidelberg.}
\bibitem{bib:openssl}{Brumley, D. and Boneh, D., 2005. Remote timing attacks are practical. Computer Networks, 48(5), pp.701-716.}
\bibitem{bib:boreale}{Boreale, M., 2003. Note per il corso di Sicurezza delle Reti.}
\bibitem{bib:montgomery}{Montgomery, P.L., 1985. Modular multiplication without trial division. Mathematics of computation, 44(170), pp.519-521.}
\bibitem{bib:sliding}{Menezes, A.J., Van Oorschot, P.C. and Vanstone, S.A., 2018. Handbook of applied cryptography. CRC press.}
\bibitem{bib:schindler}{Schindler, W., 2000, August. A timing attack against RSA with the chinese remainder theorem. In International Workshop on Cryptographic Hardware and Embedded Systems (pp. 109-124). Springer, Berlin, Heidelberg.}
\bibitem{bib:coppersmith}{Coppersmith, D., 1997. Small solutions to polynomial equations, and low exponent RSA vulnerabilities. Journal of cryptology, 10(4), pp.233-260.}

\end{thebibliography}
\end{frame}
\endgroup
%---------------------------------------------------------------------------------------
%---------------------------------------------------------------------------------------
%---------------------------------------------------------------------------------------
%---------------------------------------------------------------------------------------
%---------------------------------------------------------------------------------------
\appendix
%---------------------------------------------------------------------------------------
%---------------------------------------------------------------------------------------
%---------------------------------------------------------------------------------------
%---------------------------------------------------------------------------------------
%---------------------------------------------------------------------------------------
\begin{frame}[c,noframenumbering]

  \Huge Appendix

\end{frame}
%---------------------------------------------------------------------------------------
\begin{frame}
\frametitle{Appendix - Variance estimation in Kocher's attack}

The quantity $\Var(T - T')$ can be estimated with the formula $$\
  \frac{1}{s - 1}\sum_{j = 1}^s \left( \
    (T_j - T'_j) - \frac{1}{s}\sum_{j = 1}^s (T_j - T'_j) \
  \right)^2, \
$$
recalling that:
\begin{itemize}
  \item $s$ is the number of generated ciphertexts;
  \item $T_j$ is the sum of the time required to decrypt the $j$-th ciphertext;
  \item $T'_j = e + \sum_{i = 0}^{b - 1} t'_i, \hspace{1cm} \forall j \in \{ 1, \dots, s\}$;
  \item $e$ is the overhead time required by the attacker custom decryption;
  \item $b - 1$ is the index of the guessed bit.
\end{itemize}

\end{frame}
%---------------------------------------------------------------------------------------
\begin{frame}
\frametitle{Probability of a correct guess in Kocher's attack (1/5)}

An attacker is performing the $b$-th iteration of the Kocher's attack.

We can estimate the probability that $d^*_b$ is correct.

Suppose the attacker:
\begin{itemize}
  \item knows the first $b - 1$ bits of the private exponent $x$;
  \item can measure $T' = \sum_{i = 0}^{b - 1} t'_i$ for each ciphertext $y_j$, with $j \in \{ 1, \dots, s \}$;
\end{itemize}
If $x_b$ is correct, $T - T'$ yields $e + \sum_{i = 0}^{k - 1} t_i - \sum_{i = 0}^{b - 1} t_i = e + \sum_{i = b}^{k - 1} t_i$.

\end{frame}
%---------------------------------------------------------------------------------------
\begin{frame}
\frametitle{Probability of a correct guess in Kocher's attack (2/5)}

Now, assume that:
\begin{itemize}
  \item all the time measurements i.i.d. as $\mathcal{N}(0, 1)$;
  \item $\Var(T_j - T'_j) = \Var(e + \sum_{i = b}^{k - 1} t_i)$ for each ciphertext $y_j$;
  \item the expected variance among all ciphertexts: $\Var(e) + (k - b)\nu$, with $\nu := \Var(t_i)\ \forall i$.
\end{itemize}
However, if only the first $c < b$ bits of the exponent guess are correct, the expected variance will be $\Var(e) + (k + b - 2c)\nu$.

\end{frame}
%---------------------------------------------------------------------------------------
\begin{frame}
\frametitle{Probability of a correct guess in Kocher's attack (3/5)}

Finally, assuming $\Var(e)$ negligible, we can state that the following two probabilities are the same:

\begin{enumerate}
  \item that subtracting a correct $t'_b$ from each ciphertext will reduce the total variance more than subtracting an incorrect $t'_b$;
  \item that $d^*_b$ is correct.
\end{enumerate}

In the next two slides, we will show formulas to attain the first probability.
\end{frame}
%---------------------------------------------------------------------------------------
\begin{frame}
\frametitle{Probability of a correct guess in Kocher's attack (4/5)}

{\scriptsize
\begin{gather*}
Pr \left[
  \frac{1}{s - 1}
  \sum\limits_{j = 1}^{s} \left(
    \sqrt{k - b} X_j  + \sqrt{2(b - c)} Y_j - 0
  \right)^2
  > \frac{1}{s - 1}\sum\limits_{j = 1}^{s} \left(
    \sqrt{k - b} X_j - 0
  \right)^2
\right] \\
= Pr \left[
  (k - b) \sum\limits_{j = 1}^{s} X_j^2
  + 2(b - c) \sum\limits_{j = 1}^{s} Y_j^2
  + \sqrt{2(b - c)(k - b)} \sum\limits_{j = 1}^{s} X_j Y_j
  > (k - b) \sum\limits_{j = 1}^{s} X_j^2
\right] \\
= Pr \left[
  2(b - c) \sum\limits_{j = 1}^{s} Y_j^2
  + \sqrt{2(b - c)}\sqrt{k - b} \sum\limits_{j = 1}^{s} X_j Y_j
  > 0
\right] \\
= Pr \left[
  2 \sqrt{ 2(b - c)(k - b) } \sum\limits_{j = 1}^{s} X_j Y_j
  + 2(b - c) \sum\limits_{j = 1}^{s} Y_j^2 > 0
\right]
\end{gather*}
}
where {\scriptsize $X \sim \mathcal{N}(0, 1)$ } and {\scriptsize $Y \sim \mathcal{N}(0, 1)$ }.
\end{frame}
%---------------------------------------------------------------------------------------
\begin{frame}
\frametitle{Probability of a correct guess in Kocher's attack (5/5)}
Moreover, for $s$ large enough:
{\scriptsize
  \begin{itemize}
    \item $\sum_{j = 1}^{s}Y_j^2 \approx s$
    \item $\sum_{j = 1}^{s}X_jY_j \sim \mathcal{N}(0, \sqrt{s})$,
  \end{itemize}
}
yielding

{\scriptsize
  \begin{align*}
  Pr \left(
    2\sqrt{2(b - c)(k - b)} \left( \sqrt{s} Z \right)
      + 2(b - c)s > 0
  \right) &= Pr \left(
    Z > - \frac{ \sqrt{s(b - c)} }{2(k - b)}
  \right) \\
  &= Pr \left(
    Z < \frac{\sqrt{s(b - c)}}{2(k - b)}
  \right) \\
  &= \Phi \left( \sqrt{ \frac{s(b - c)}{2(k - b)} } \right)
  \end{align*}
}

where {\scriptsize $\Phi(x)$ } is the cumulative density function of {\scriptsize $\mathcal{N}(0, 1)$ }.

\end{frame}
%---------------------------------------------------------------------------------------

\end{document}

