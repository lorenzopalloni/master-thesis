\documentclass[12pt]{article}
\usepackage[italian,english]{babel}
\usepackage[utf8x]{inputenc} %eventualmente da cambiare se ci sono problemi con accenti
\usepackage{hyperref}
\usepackage[normalem]{ulem}
\useunder{\uline}{\ul}{}

\setlength{\textwidth}{16 cm}
\setlength{\oddsidemargin}{0 cm}
\setlength{\topmargin}{-1.5 cm}
\setlength{\textheight}{30 cm}

\title{Optimization Techniques of Deep Learning Models for Visual Quality Improvement}

\begin{document}

\section*{CONTENTS}

\begin{description}
    \item[1] Introduction \hfill 5
    \item[2] Background \hfill 7
    \item[3] Metrics \hfill 9
    \begin{description}
        \item[3.1] Traditional Metrics \hfill 9
        \item[3.2] Perceptual Metrics \hfill 10
        \item[3.3] No-Reference Metrics \hfill 11
        \item[3.4] Video Quality Metrics \hfill 12
    \end{description}
    \item[4] Architectures \hfill 15
    \begin{description}
        \item[4.1] UNet Architecture \hfill 15
        \item[4.2] SRUNet Architecture \hfill 17
        \item[4.3] Training Setup \hfill 17
    \end{description}
    \item[5] Optimizations \hfill 23
    \begin{description}
        \item[5.1] Quantization \hfill 24
        \item[5.2] TensorRT to speed up inference \hfill 25
        \item[5.3] Data-loader to speed up training \hfill 27
    \end{description}
    \item[6] Experiments \hfill 29
    \begin{description}
        \item[6.1] Quantitative Results \hfill 29
        \item[6.2] Qualitative Results \hfill 33
    \end{description}
    \item[7] Conclusions \hfill 41
    \item[8] Acknowledgements \hfill 43
\end{description}

\end{document}
